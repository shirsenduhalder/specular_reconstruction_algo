%%%%%%%%%%%%%%%%%%%%%%%%%%%%%%%%%%%%%%%%%%%%%%%%%%%%%%%%%%%%%%%%%%%%%
% LaTeX Template: Project Titlepage Modified (v 0.1) by rcx
%
% Original Source: http://www.howtotex.com
% Date: February 2014
% 
% This is a title page template which be used for articles & reports.
% 
% This is the modified version of the original Latex template from
% aforementioned website.
% 
%%%%%%%%%%%%%%%%%%%%%%%%%%%%%%%%%%%%%%%%%%%%%%%%%%%%%%%%%%%%%%%%%%%%%%

\documentclass[12pt]{article}
\usepackage[a4paper]{geometry}
\usepackage[myheadings]{fullpage}
\usepackage{fancyhdr}
\usepackage{lastpage}
\usepackage{graphicx, wrapfig, subcaption, setspace, booktabs}
\usepackage[T1]{fontenc}
\usepackage[font=small, labelfont=bf]{caption}
\usepackage{tocloft}
\usepackage[protrusion=true, expansion=true]{microtype}
\usepackage[english]{babel}
\usepackage{sectsty}
\usepackage{url, lipsum}
\usepackage{tgbonum}
\usepackage{gensymb}
\usepackage{hyperref}
\usepackage{xcolor}

\newcommand{\HRule}[1]{\rule{\linewidth}{#1}}
\newcommand{\etal}{\textit{et al. }}
\newcommand{\hardware}{\texttt{Hardware requirements: }}
\newcommand{\dagg}{\textsuperscript{\color{blue}$\mathbf{\ddagger}$}}
\onehalfspacing
\setcounter{tocdepth}{5}
\setcounter{secnumdepth}{5}



%-------------------------------------------------------------------------------
% HEADER & FOOTER
%-------------------------------------------------------------------------------
%\pagestyle{fancy}
%\fancyhf{}
%\setlength\headheight{15pt}
%\fancyhead[L]{Student ID: 1034511}
%\fancyhead[R]{Anglia Ruskin University}
%\fancyfoot[R]{Page \thepage\ of \pageref{LastPage}}
%-------------------------------------------------------------------------------
% TITLE PAGE
%-------------------------------------------------------------------------------

\begin{document}
{\fontfamily{cmr}\selectfont
\title{ \normalsize \textsc{}
		\\ [2.0cm]
		\HRule{0.5pt} \\
		\LARGE \textbf{\uppercase{Algorithms for Specular Object Reconstruction}
		\HRule{2pt} \\ [0.5cm]
		\normalsize \today \vspace*{5\baselineskip}}
		}

\date{}


\maketitle
\newpage
\tableofcontents
\newpage

%-------------------------------------------------------------------------------
% Section title formatting
\sectionfont{\scshape}
%-------------------------------------------------------------------------------

%-------------------------------------------------------------------------------
% BODY
%-------------------------------------------------------------------------------

\section{Notes}
\textit{Three-dimensional scanning of specular and diffuse
metallic surfaces using an infrared technique} is a good paper for previous work.
Check out \href{https://mynameismjp.wordpress.com/2016/10/09/sg-series-part-1-a-brief-and-incomplete-history-of-baked-lighting-representations/}{this} for a tutorial on light representation using spherical gaussians.
\dagg means checked all the citations of the paper.
\section{Shape from Specular Flow}
\input{specular_flow.tex}
%\newpage
%\section{Getting Started}
%\input{getstart.tex}


\section{Shape from distortion}
Shape from distortion are a group of methods for specular surface reconstruction that makes use of observations of known or unknown patterns that are distorted by specular reflection. These techniques usually assume perfect, mirror-like surface reflectance.

\dagg Rehab \etal impose a specularity constraint by defining a matching cost function for two different camera views. The matching cost of the disparity is based on differences on normal estimates rather than intensities. Cost aggregation is performed using anisotropic diffusion of the cost function.

\hardware Pair of cameras and an source which emits temporally encoded light pattern with a dense set of calibrated light sources. 

\dagg Yamazaki \etal \cite{yamazaki20103d} use active binocular stereo using a projector for diffuse surfaces and a display for specular surfaces. They use a weighted combination of the matching cost functions of projector-display for glossy surfaces. Stereo matching cost function for projector is based on the minimizing the phase difference between candidate matching pixels while for the display is based on minimizing the absolute phase differences between display and the left and right camera points along epipolar lines. 

\hardware 2 Nikon CCD Cameras, a projector and an Apple Cinema HD Display.

\dagg Liu \etal \cite{liu2013mirror} solves the problem of reconstructing the shape of a smooth mirror-like surface when dense correspondences from a static reference target are present and prove the uniqueness of this solution. They take it a step forward by constructing surfaces with sparse correspondences by solving a non-linear optimization problem using least-squares. 

\dagg Kohler \etal \cite{kohler2013full} present Orbital Camera (OrcaM), a spherical device for simultaneous acquisition of geometrical, color and reflectance properties. It contains a glass carrier for inspection of the bottom part.  The actual camera obtained by the combination of a perspective camera and refractive layers of axial type. A point cloud reconstructed with OrcaM usually consists of 200-300 million oriented points. For size efficiency, the coarse object shape is represented by a low resolution mesh while the fine details are stored in a normal map.

\hardware 7 Canon EOS 500D DSLR, Sanyp PLV Z4000 (projector), 633 LEDs at an angular distance of $7.5 \degree$.

\dagg Weinmann \etal \cite{weinmann2013multi} employ a turntablebased setup with several cameras and displays that are used to display illumination patterns which are reflected by the object surface. Via a non-parametric clustering of normal hypotheses derived for each point in the scene both the most likely local surface normal and a local surface consistency estimate is obtained which is used to itertively construct the surface.

\hardware 11 cameras, 2 displays

\dagg Rozenfeld \etal \cite{rozenfeld2010dense} recover the 3D shape of mirror like objects by capturing the reflection of stripes by a camera. The direction of the displayed stripes and their reflection in the image are related by a 1D homography matrix. They focus on sparse correspondences from which the depth and the local shape are estimated at sub-millimeter accuracy.

\dagg Nguyen \etal \cite{nguyen2019single} propose a single-shot defectometry for real-time measurement of 3D surface profle using a single composite pattern by the addition of orthogonal fringe patterns. They adapt the Fourier Transform (FT) method and the spatial carrier frequency phase-shifting (SCPS) technique for accurate phase estimation. They also demonstrate measuring surface even under harsh environmental conditions (vibrations)
in real-time.

\dagg Balzer \etal \cite{balzer2014cavlectometry} use the deflectometry principle for reconstruction of complex and large specular objects. They perform deflectometry in an Atomatic Virtual Environment called CAVE from any vantage point.

\hardware A cube like structure which they call CAVE. \\

If required, also check:

\begin{enumerate}
	\item \dagg Han \etal \cite{han2016mirror} construct surfaces based on observing the reflections of a moving reference plane on the mirror surface.
\end{enumerate}

\section{Shape from specularities/reflectance}
Shape from specularity approaches rely on the observation of surface highlights caused by specular reflection at some surface points. Standard stereo techniques lead to errors in depth estimation.

\dagg Tappen \cite{tappen2011recovering} was the first where shape from single image was recovered under unknown illumination and included simple constraints on the surface curves. 

\dagg Sankaranarayanan \etal \cite{sankaranarayanan2010specular} develop a relationship between reflection correspondences (RCs) for specular reconstruction under finite motion. Any two points in one or more images of a specular surface which observe the same environmental feature are denoted as RCs. This was the first method to reconstruct mirror surfaces using sparse RCs. 

\dagg Chen \etal \cite{chen2017microfacet} devise a novel analytical microfacetbased isotropic reflectance model from the ellipsoid normal distribution function. We also introduce a physically interpretable approximate of our model that is particularly serviceable for specular reflectance analysis. 

\dagg Nam \etal \cite{nam2018practical} acquires SVBRDF information of 3D objects using a single camera, not limited to planar surfaces, and not requiring a commercial 3D scanner to accurately capture input geometry. Reconstruction of 3D objects takes SVBRDF information into account instead of assuming diffuse surfaces. Involves finding spatially-varying reflectance and normals and imposing photometric constraints while updating geometry.

\dagg Nayar and Gupta \cite{nayar2012diffuse} show that structured light has two limitations - strong highlights for specular reflections and effects of shadows. They establish that diffusion of the patterns along axis of translation can mitigate the adverse effects of specularities and shadows.

\hardware AAXA M2 Projector, Canon XSi SLR, lenticular array (Edmund Optics part number NT43-029) as diffuser. 

\dagg Zheng \etal \cite{zheng2000acquiring} recover 3D models of rotating objects using continuous images.  Circular-shaped light sources that generate conic rays are used to illuminate the rotating object in such a way that highlighted stripes can be observed on most of the specular surfaces. Surface shapes can be computed from the motions of highlights in the continuous images by using either specular motion stereo or single specular trace mode.

\dagg Li \etal \cite{li2018learning} recover non-Lambertian, spatially-varying BRDFs and complex geometry belonging to any arbitrary shape class, from a single RGB image captured under a combination of unknown environment illumination and
flash lighting using a deep neural network to regress shape and reflectance. They also propose a dataset.


\dagg Xia \etal \cite{xia2016recovering} and Dong \etal \cite{dong2014appearance} are from the same research group and estimate spatially-varying surface reflectance and geometry of \textit{rotating objects} under unknown static illumination. 

\dagg Ngo \etal \cite{ngo2019reflectance} recovers the isotropic reflectance and depth map under natural illumination using a light field camera and a 360-degree camera to capture the illumination. They employ a multi-stage algorithm where depth is estimated first using plane sweeping, followed by depth and normal estimation with Lambertian and the final stage incoroporates updating depth, normal and reflectance.

\hardware Lytro ILLUM camera for light field and Theta S for natural illumination. Refer for datasets used. 

Godard \etal \cite{godard2015multi} build a visual hull of the object and estimate the surface normal using a probability distribution under a Bayesian framework based on correspondences between the colours reflected from the object and those in the environment. They also incorporate inter reflections into this probabilistic surface normal estimate and iteratively construct the object. 

Wang \etal \cite{wang2016svbrdf} use light-field cameras and derive a spatially-varying (SV)BRDF-invariant theory for recovering 3D shape and reflectance for upto 2-lobes once shape is retrieved.

ZHou \etal \cite{zhou2019non} introduce a concentric multi-spectral light field design that recovers the shape and reflectance of surfaces with arbitrary materials. The system consists fo an array of cameras arranged on concentric circles where each ring captures a specific spectrum. The Phong dichromatic reflectance model is used to estimate surface shape and reflectances.

\hardware Point Grey GS3-U3-51S5MC monochrome camera, Tunable liquid crystal spectral filter (KURIOUS - WL1) 

\textit{Refer if micro-geometry involved: Johnson \etal \cite{johnson2011microgeometry}. Refer Netz \etal \cite{netz2012recognition} if lambertian + specular. Refer Hui \etal or Riviere \etal \cite{hui2017reflectance, riviere2016mobile} if near-planar materials are considered.}


\section{Direct Ray/Slice measurements}
\dagg Tin \etal \cite{tin20163d} adopt a two-layer liquid crystal display (LCD) setup to encode the illumination directions and devise an efficient ray coding scheme by only considering the useful rays. To recover the mirror-type surface, they derive a normal integration scheme under the perspective camera model. They further develop a surface reconstruction algorithm under perspective projection to generate complete profile of a mirror-type object.

\textit{Ye \etal \cite{ye2012angular} is based on reconstruction in dynamic fluids.}

\section{Polarization analysis} 
\dagg Lu \etal \cite{lu2019mirror} use a dense illumination field with angularly varying polarization states, called polarization field. They use a commercial LCD and
mathematically model the liquid crystals as polarization rotators using Jones calculus and show that the rotated polarization states of outgoing rays encode angular information. A reflection image formation model based on the Fresnel's equations is derived and the incident ray positions and directions are estimated using the polarization field. 

Jospin \etal \cite{jospin2018embedded} incorporate polarization filters in camera designs to separate the specular and diffuse components.

\dagg Ba \etal \cite{ba2019physics} use deep learning techniques to solve the Shape from Polarization problem by blending approximate physics into neural networks. 

\section{Other techniques}
\textbf{Scanning from heating:}  Unlike classical active triangulation
approaches, the principle of this technique, called scanning
from heating (SFH) is based on the measurement of the IR
radiation that is emitted by the object instead of the reflection
of visible radiation. A laser source is used to cause a local
elevation of temperature; the 3-D coordinates of each point
are then extracted from the IR images with a prior geometric
calibration of the system. Bajard \etal \cite{bajard2012three} implement a 3-D scanner prototype, based on the SFH technique. The system consists of a diode e laser emitting at $\lambda = 808 nm$, and an IR imaging detector, which is sensitive to mid-wavelength IR. They explain the physical properties considered for the heating process and a finite element model to choose the system settings. The accuracy is in-dependant of the smoothness of the material (specular/rough). 


Sankaranarayanan \etal \cite{veeraraghavan2010image} state some properties of images of smooth surfaces with mirror-like reflectance that is invariant to change in camera parameters, illumination, etc.

Shroff \etal \cite{shroff2011finding} show that regions of high curvature in highly
specular objects produce specular reflections largely independent of the location of the illumination sources and an inexpensive multi-flash camera can be used to reliably detect such features. \textit{Maybe useful for the chip-noise modelling}

Lombardi \etal \cite{lombardi2016radiometric} recovers the shape and reflectance on non-lambertian surfaces. Can be useful as we have depth data. 

-A Lightweight Approach for On-the-Fly Reflectance Estimation

\section{BRDFs and Dataset papers}
Shi \etal \cite{shi2016benchmark} survey and categorize existing methods using a photometric stereo taxonomy emphasizing on non-Lambertian and uncalibrated methods. They introduce a photometric stereo dataset called \textbf{DiLiGenT} for non-Lambertian materials. This paper contains citations of some literature that handles anisotropic BRDFs. Google scholar: anisotropic brdf estimation.

\dagg Wang \etal \cite{wang2008modeling} represent anisotropic reflectance using a microfacet-based BRDF which tabulates the facet's normal distribution as a function of the surface location. 

\dagg Xu \etal \cite{xu2013anisotropic} present an anisotropic Spherical Gaussian (ASG) function which is much more effective and efficient in representing anisotropic spherical functions than Spherical Gaussians (SGs). ASGs are rotationally invariant and capable of representing all-frequency signals and have approximate closed-form solutions for their integral, product and convolution operators, whose errors are nearly negligible.

\dagg Yan \etal \cite{yan2016position} treat specular surface as a 4D position-normal distribution and fir this distribution using millions of 4D Gaussians which leads to closed-form solutions for the BRDF. Can be used for anisotropic BRDFs.

\dagg Yan \etal \cite{yan2018rendering} is a rendering based paper for rendering specular reflection. Results show both single-wavelength and spectral solutions to reflection from common everyday objects, such as brushed(exhibits anisotropy), scratched and bumpy metals.

\dagg Raymond \etal \cite{raymond2016multi} is again a rendering paper tailored to the high angular and spatial frequencies exhibited by scratched materials. Their SVBRDF reproduces directional highlights/patterns. \texttt{Check cited papers for more anisotropic BRDFs}

\dagg Lee \etal \cite{lee2018practical} revisit the traditional V-groove cavity microfaucet model and derive an analytical, cost-effective solution for multiple scattering in rough surfaces. Their model is made up of both real and virtual grooves, and allows us to calculate higher-order scattering in the microfacets
in an analytical fashion. They extend our model to include nonsymmetric grooves, allowing for additional degrees of freedom on the surface geometry, improving multiple reflections at grazing angles.

\dagg Heitz \etal \cite{heitz2015sggx} propose a novel microflake representation to represent isotropic and anisotropic microstructures and use it for multi-scale rendering. Might be useful in noise modelling. 

\dagg Chen \etal \cite{chen2014reflectance} generalize linear light source reflectometry by modulating the intensity along the linear light source, and show that a constant and two sinusoidal lighting patterns are sufficient for estimating the local shading frame and anisotropic surface reflectance. 

\dagg Guarnera \etal \cite{guarnera2016brdf} presents an overview of BRDF (Bidirectional Reflectance Distribution Function) models used to represent surface/material reflection characteristics, and describes current acquisition methods for the capture and rendering of photorealistic materials.

\dagg Tunwattanapong \etal \cite{tunwattanapong2013acquiring} use spherical harmonic illumination to separate out the diffuse and specular component. They also use higher order spherical harmonics to measure albedo, refelction vector, roughness and anisotropic parameters of a specular reflection lobe. 

\dagg Li \etal \cite{li2017robust} propose a variational energy minimization framework for robust recovery of shape in multiview stereo with complex, unknown
BRDFs. This framework handles non-Lambertian effects, noise and non-differential motions without restrictive priors. They demonstrate recovering shape from light fields using the same.

\dagg Gutsche \etal's \cite{gutsche2017surface} method works with multiple views and a fixed light source unlike photometric stereo.They solve a non-linear optimization problem to infer BRDF and surface normals simultaneously by exploiting densely sampled light fields. 


\section{Time of flight Sensors}
Jarabo \etal \cite{jarabo2017recent} provides a brief insight into transient imaging using time of flight sensors till 2016.
%-------------------------------------------------------------------------------
% REFERENCES
%-------------------------------------------------------------------------------
\newpage
\bibliographystyle{ieeetr}
\bibliography{refs.bib}
%[2]John W. Eaton, David Bateman, Sren Hauberg, Rik Wehbring (2015). GNU
%Octave version 4.0.0 manual: a high-level interactive language for numer-
%ical computations. Available: http://www.gnu.org/software/octave/doc/
%interpreter/. 
}
\end{document}

%-------------------------------------------------------------------------------
% SNIPPETS
%-------------------------------------------------------------------------------

%\begin{figure}[!ht]
%	\centering
%	\includegraphics[width=0.8\textwidth]{file_name}
%	\caption{}
%	\centering
%	\label{label:file_name}
%\end{figure}

%\begin{figure}[!ht]
%	\centering
%	\includegraphics[width=0.8\textwidth]{graph}
%	\caption{Blood pressure ranges and associated level of hypertension (American Heart Association, 2013).}
%	\centering
%	\label{label:graph}
%\end{figure}

%\begin{wrapfigure}{r}{0.30\textwidth}
%	\vspace{-40pt}
%	\begin{center}
%		\includegraphics[width=0.29\textwidth]{file_name}
%	\end{center}
%	\vspace{-20pt}
%	\caption{}
%	\label{label:file_name}
%\end{wrapfigure}

%\begin{wrapfigure}{r}{0.45\textwidth}
%	\begin{center}
%		\includegraphics[width=0.29\textwidth]{manometer}
%	\end{center}
%	\caption{Aneroid sphygmomanometer with stethoscope (Medicalexpo, 2012).}
%	\label{label:manometer}
%\end{wrapfigure}

%\begin{table}[!ht]\footnotesize
%	\centering
%	\begin{tabular}{cccccc}
%	\toprule
%	\multicolumn{2}{c} {Pearson's correlation test} & \multicolumn{4}{c} {Independent t-test} \\
%	\midrule	
%	\multicolumn{2}{c} {Gender} & \multicolumn{2}{c} {Activity level} & \multicolumn{2}{c} {Gender} \\
%	\midrule
%	Males & Females & 1st level & 6th level & Males & Females \\
%	\midrule
%	\multicolumn{2}{c} {BMI vs. SP} & \multicolumn{2}{c} {Systolic pressure} & \multicolumn{2}{c} {Systolic Pressure} \\
%	\multicolumn{2}{c} {BMI vs. DP} & \multicolumn{2}{c} {Diastolic pressure} & \multicolumn{2}{c} {Diastolic pressure} \\
%	\multicolumn{2}{c} {BMI vs. MAP} & \multicolumn{2}{c} {MAP} & \multicolumn{2}{c} {MAP} \\
%	\multicolumn{2}{c} {W:H ratio vs. SP} & \multicolumn{2}{c} {BMI} & \multicolumn{2}{c} {BMI} \\
%	\multicolumn{2}{c} {W:H ratio vs. DP} & \multicolumn{2}{c} {W:H ratio} & \multicolumn{2}{c} {W:H ratio} \\
%	\multicolumn{2}{c} {W:H ratio vs. MAP} & \multicolumn{2}{c} {\% Body fat} & \multicolumn{2}{c} {\% Body fat} \\
%	\multicolumn{2}{c} {} & \multicolumn{2}{c} {Height} & \multicolumn{2}{c} {Height} \\
%	\multicolumn{2}{c} {} & \multicolumn{2}{c} {Weight} & \multicolumn{2}{c} {Weight} \\
%	\multicolumn{2}{c} {} & \multicolumn{2}{c} {Heart rate} & \multicolumn{2}{c} {Heart rate} \\
%	\bottomrule
%	\end{tabular}
%	\caption{Parameters that were analysed and related statistical test performed for current study. BMI - body mass index; SP - systolic pressure; DP - diastolic pressure; MAP - mean arterial pressure; W:H ratio - waist to hip ratio.}
%	\label{label:tests}
%\end{table}%\documentclass{article}
%\usepackage[utf8]{inputenc}

%\title{Weekly Report template}
%\author{gandhalijuvekar }
%\date{January 2019}

%\begin{document}

%\maketitle

%\section{Introduction}

%\end{document}
