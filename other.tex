\textbf{Scanning from heating:}  Unlike classical active triangulation
approaches, the principle of this technique, called scanning
from heating (SFH) is based on the measurement of the IR
radiation that is emitted by the object instead of the reflection
of visible radiation. A laser source is used to cause a local
elevation of temperature; the 3-D coordinates of each point
are then extracted from the IR images with a prior geometric
calibration of the system. Bajard \etal \cite{bajard2012three} implement a 3-D scanner prototype, based on the SFH technique. The system consists of a diode e laser emitting at $\lambda = 808 nm$, and an IR imaging detector, which is sensitive to mid-wavelength IR. They explain the physical properties considered for the heating process and a finite element model to choose the system settings. The accuracy is in-dependant of the smoothness of the material (specular/rough). 


Sankaranarayanan \etal \cite{veeraraghavan2010image} state some properties of images of smooth surfaces with mirror-like reflectance that is invariant to change in camera parameters, illumination, etc.

Shroff \etal \cite{shroff2011finding} show that regions of high curvature in highly
specular objects produce specular reflections largely independent of the location of the illumination sources and an inexpensive multi-flash camera can be used to reliably detect such features. \textit{Maybe useful for the chip-noise modelling}

Lombardi \etal \cite{lombardi2016radiometric} recovers the shape and reflectance on non-lambertian surfaces. Can be useful as we have depth data. 

-A Lightweight Approach for On-the-Fly Reflectance Estimation